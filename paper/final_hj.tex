\def\year{2016}\relax
%File: formatting-instruction.tex
\documentclass[letterpaper]{article}
\usepackage{aaai16}
\usepackage{times}
\usepackage{helvet}
\usepackage{courier}
\usepackage{graphicx}
\usepackage{bbm}
%\usepackage{graphics}
\usepackage{subcaption}
\usepackage{mathptmx} % assumes new font selection scheme installed
\usepackage{amsmath} % assumes amsmath package installed
\usepackage{amssymb}  % assumes amsmath package installed
\usepackage{tabularx, adjustbox}
\newcommand{\norm}[1]{\left\lVert#1\right\rVert}
\newcommand{\argmax}{\operatornamewithlimits{argmax}}
\newcommand{\argmin}{\operatornamewithlimits{argmin}}
\newcommand{\real}{\mathbb{R}}

\frenchspacing
\setlength{\pdfpagewidth}{8.5in}
\setlength{\pdfpageheight}{11in}
\pdfinfo{
/Title (Deep Reinforcement Learning with Hierarchical Recurrent Encoder-Decoder for Conversation)
/Author (Heejin Jeong, Xiao Ling)}
\setcounter{secnumdepth}{0}  
 \begin{document}
% The file aaai.sty is the style file for AAAI Press 
% proceedings, working notes, and technical reports.
%
\title{Deep Reinforcement Learning with Hierarchical Recurrent Encoder-Decoder for Conversation }
\author{Heejing Jeong \and Xiao Ling\\
  {\tt [heejinj, lingxiao]@seas.upenn.edu}}
\maketitle

\section{Introduction}
In recent years, numerous neural network based methods have been introduced for automatic dialog generation. Sutskever et al. presented a sequence to sequence learning model with deep neural networks (SEQ2SEQ) consisting of two multi-layered Long Short-Term Memory (LSTM) and showed that it outperformed a standard SMT-based system on an English to French translation task \cite{Sutskever}. Vinyals et al. applied this sequence to sequence framework to conversational modeling and their model predicted the next sentence given the previous sentence \cite{Vinyals}. Unlike other models that had been used widely, this model achieved better performance requiring much fewer hand-crafted rules. \\
However, since the SEQ2SEQ model was originally proposed for machine translation tasks, it does not capture previous conversations when it is applied to a conversation task. Being able to address previously mentioned topics or information is essential in conversation. In order to solve this issue, Serban et al. extended the hierarchical recurrent encoder-decoder (HRED), proposed by Sordoni et al. \cite{Sordoni}, to the dialog domain \cite{Serban}. Although they used only triple utterances for implementation, they were able to show their proposed model outperformed both $n$-gram based models and baseline neural network models. Another issue of the SEQ2SEQ model is that it tends to be short-sighted since the model does not consider its future outcomes. Addressing this issue, Li et al. proposed a novel approach for a dialog generation task combining a policy gradient optimization method and the SEQ2SEQ model (DRL-SEQ2SEQ) \cite{Li}. The policy gradient optimization method is one of policy-based methods in Reinforcement Learning (RL), and it updates its policy parameters in order to maximize a learner's expected discounted future total reward. Thus, we can expect the model would generate an outcome more likely to continue the current conversation. \\
While the HRED model does not consider its future outcomes, the DRL-SEQ2SEQ model cannot address its conversation history other than a few immediate previous utterances. In this project, we studied the HRED model for dialog generation in open domain \cite{Serban} and DRL-SEQ2SEQ. We initially aimed to replace SEQ2SEQ model in DRL-SEQ2SEQ with HRED in order to utilize the advantage of each model. However, due to the time constraint, we were not able to combine both models but tried implementing them on OpenSubtitle Dataset. 

\section{Hierarchical Recurrent Encoder Decoder}
In this section explain the hiearchical recurrent neural net (HRED) model proposed by \cite{Serban} and \cite{Sordoni}, the problem they posed and how HRED solves these problems. In the original paper by \cite{Sordoni}, HRED was proposed as a model that could generate context-senstitive queries, then \cite{Serban} adapted it to train an end-to-end history-aware conversation model.

\subsection{Recurrent Neural Net}

Now we wish to review recurrent neural net (RNN) in some detail. We denote a dialogue as a sequence of $M$ utterances $U_m$: $D = \{U_1, \ldots, U_M\}$ between two speakers, each utterance contains $N_m$ tokens: $U_m = \{w_{m,1},\ldots,w_{m,N_m}\}$. $\cite{Serban}$ allowed the random variable $w_{m,n}$ to range over the vocabular and ``speech acts", although in our case we only considered words. Next they defined a distribution $P$ with parameter $\theta$ over the set of all possible dialogues of any length, and factorized $P$ by:
    \begin{align*}
        P_{\theta}(U_1, \ldots, U_M) &= \prod_{m=1}^M P_{\theta}(U_m | U_{<m}) \\
                                     &= \prod_{m=1}^M \prod_{n=1}^{N_m} P_{\theta} (w_{m,n}|w_{m,<n}, U_{<m}),
    \end{align*}

where $U_{<m} = \{U_1, \ldots, U_{m-1}\}$. In other words, the conditional probability of the current word is only a function of previous words in the utterance and previous utterances. Next, \cite{Serban} represents $P_{\theta}$ with:
    \begin{align*}
        P_{\theta} (w_{n+1} = v | w_{\leq n}) = \frac{\exp(g(h_n,v))}{\sum_{v'} \exp(g(h_n, v'))},
    \end{align*}

where $h_n \in \mathbbm{R}^{d_h}$ is a hidden state computed by a recurrent neural net:

    \begin{align*}
        &h_n = tanh (H h_{n-1} + I_{w_n}) \\
        &g(h_n, v) = O^{T}_{w_n} h_n,
    \end{align*}

where $I \in \mathbbm{R}^{d_h \times |V|}$ is the input word embeding. Again note that the hidden state is only a function of the past. 

\subsection{Hierarchical Recurrent Neural Net}

HRED augments the basic RNN model by explicitly learning a transition function over the hidden dynamic of the conversation or  ``session". Serban and Sordoni learned this transition function using a session level RNN. Specifcially, this RNN predicts the next utterance by:
    \begin{align*}
        P(U_m|U_{<m}) &= \prod_{n=1}^{N_m} P(w_n | w_{<n-1}, U_{m-1}) \\
                      &= \frac{\exp(o_{v}^T w(d_{m,n-1}, w_{{m,n-1}}))}{\sum_{k} \exp(o_{k}^Tw(d_{m,n-1}, w_{m,n-1}))},
    \end{align*}

where $w(d_{m,n-1}, w_{m,n-1}) = H_o d_{m,n-1} + E_o w_{m,n-1} + b_o,$ so that the next utterance is a function of all pervious utterances. In summary, in HRED each utterance is a function of the previous utterance, and each word in the utterance is a function of all previous words in the utterance. A session level RNN learns the transition between utterances vector, and sentence level RNN learns the transition between words. 

\subsection{Learning}

HRED is trained end-to-end by maximizing the log-likelihood of the whole session $S$:

    \begin{align*}
        Loss(S) &= \sum_{m=1}^M log P(U_m|U_{<m}) \\
                &= \sum_{m=1}^M \sum_{n=1}^{N_m} log P(w_{m,n} | w_{m,<n}, U_{<m}).
    \end{align*}

We train the model by dividing the data set into sessions with four utterances each, the vocabulary size was set to be 50,005, and the maximum sentence length was 50 tokens.





































\section{Deep Reinforcement Learning for Dialog Generation}
\section{DRL with HRED}
Our model differs from DRL-SEQ2SEQ model in three aspects. First, instead of using two previous dialogue utterances to define a state, we define a state $s_t$ at time $t$ as a hidden state $c_t$ of the dialogue level RNN at the time, $s_t=c_t = f(c_{t-1},h_T^{(t)})$, where $h_T^{(t)}$ is the last hidden state of the $t$th utterance and $f$ is a parametrized non-linear function. We use the same definition for actions - generating a dialogue utterance. Thus, a RL policy is defined as $\pi(a_t|s_t) =p_{RL}(u_{t+1}|c_t)$. Second, we pre-train a model using HRED to initialize the RL policy $\pi$. Finally, we replace the pre-trained SEQ2SEQ model with the pre-trained HRED model in reward functions of RL. 

\section{Experiment}

\begin{figure}[bt!]
    \centering
    \includegraphics[width=0.48\textwidth]{mi_architecture} 
    \caption{\small SEQ2SEQ and MI model Code Architecture}
    \label{fig:mi_architecture}
 \end{figure}

\subsection{Dataset}
We first trained the modified SEQ2SEQ model on the CALLHOME American English Speech corpus (LDC97S42), consisting of 120 of 30-minute phone conversations between native English speakers.The size of vocabulary we used is 8038. However, in most of the conversation datasets, one speaker talks a long sentence or several sentences in a row and the other speaker answers with one or two words such as \textit{"mhm"},\textit{"okay"}, and \textit{"yeah"} as shown in Table.\ref{table:phone_data}. In fact, about 20\% of the utterances of the dataset is \textit{"mhm"}. As a results, the trained model after 44200 steps generated a short answer for long input utterances and a long answer for short input utterances (Table.\ref{table:phone_result}). Therefore, we concluded that this dataset may not have been appropriate for learning a dialog generation model.

Instead, we used the OpenSubtitles corpus (http://www.opensubtitles.org/) for both HRED and DRL. The OpenSubtitles corpus is a bilingual parallel corpus composed of movie subtitles. Originally constructed for machine translation, it has 20,400 files with 149.44 million tokens and 22.27 million sentence fragments. We used the 28 documents with English counterparts. The corpus is preprocessed by removing all ASCII symbols and adding consecutive speaker turns in an ad-hoc manner, that is to say we defaulted the first line in the script to be that of speaker one, the second line to be speaker two, and so on. Next, the corpus was normalized by case-folding, white space stripping, and converting all tokens to lower cases. Proper nouns and numbers were kept as is. 

with 50005 vocabulary size. The DRL paper also used this dataset, but unlike the paper, we didn't extract \textit{"i don't know what you are talking about"}. Instead, we refined abbreviated words such as \textit{"i'm"} $\rightarrow$ \textit{"i am"}, {"you're"} $\rightarrow$ \textit{"you are"}, {"i've"} $\rightarrow$ \textit{"i have"}, {"wanna"} $\rightarrow$ \textit{"want to"}, {"gonna"} $\rightarrow$ \textit{"going to"}, {"don't"} $\rightarrow$ \textit{"do not"}, etc. 

\begin{table}[t!]
    \centering
    \small
    \caption{\small Example Conversation in the CALLHOME corpus}
    \begin{tabular}{rl}
      \hline
        \textbf{A:} & who is in um someone not that they have problems but\\
        		    & someone who's like an okay student but kind of on the\\
        			& borderline you know like maybe not a great homelife\\
        			& and we would ha i got paired up\\
		\textbf{B:} & uh-huh \\
\textbf{A:} & with someone at um lipsmack i forget the school it was\\
			& actually in port richmond um breath i forget the name\\
			& of the school\\\
\textbf{B:} & really\\
\textbf{A:} & hensfiel no it was in the philadelphia school system and\\ 
			& it it was a middle school\\
\textbf{B:} & mhm\\
      \hline
    \end{tabular}
    \label{table:phone_data}
\end{table}


\begin{table}[t!]
    \centering
    \small
    \caption{\small Example Conversation in the OpenSubtitles corpus}
    \begin{tabular}{rl}
      \hline
        \textbf{A:} & a proud rebellious son who was sold to living death in \\
                    & the mines of libya before his thirteenth birthday \\
        \textbf{B:} & there under whip and chain and sun he lived out his \\
                    & youth and his young manhood dreaming the death of slavery \\
                    & 2 000 years before it finally would die \\
\textbf{A:} & back to work\\
\textbf{B:} & get up spartacus you thracian dog\\
\textbf{A:} & come on get up\\ 
\textbf{B:} & my ankle my ankle\\
      \hline
    \end{tabular}
    \label{table:phone_data}
\end{table}

\subsection{Implementation}
Although it is not common to explain code structures in an academic paper, we explicitly mentioned functions and files in tensorflow we used as well as those written by us in this section.  
For the SEQ2SEQ model, we applied Sequence to sequence with attention mechanism and used GRU cells instead of LSTM cells. We used 3 layers and 512 units for all models. We implemented models in tensorflow version 1.0.1. We used \texttt{embedding\_attention\_seq2seq()} function in \texttt{seq2seq.py} and related functions/codes written in tensorflow. We also utilized \texttt{seq2seq\_model.py} in tensorflow version 0.12.x as a bottom line.\\
The \texttt{seq2seq\_model.py} is structured as follow (see Fig.\ref{fig:mi_architecture}): when the model is initialized, three main lists of placeholders are constructed - encoder, decoder, and weight. The length of the lists are encoder size, decoder size, and decoder size, respectively. Each element of a list is a tensor with a shape [batch size, None]. Variables are also built and initialized. In each step of training, batch size data are sampled in a given training dataset and converted into a proper format in the function \texttt{get\_batch()} by operations such as padding, adding GO id, sizing, etc. Then, these are fed to the placeholders. Output logits are computed through \texttt{embedding\_attention\_seq2seq()} function and losses are computed using a softmax loss function. Then, variables in the encoder-decoder recurrent neural network (RNN) are updated. \\
The mutual information model requires a pre-trained SEQ2SEQ model and a pre-trained backward SEQ2SEQ model for its initialization and for computing mutual information score. The model first builds three placeholder lists - concatenated states (e.g. $u_{t-1},u_t$), the most recent state (e.g. $u_t$) and weights for the backward model. Instead of sampling several action candidates given an input from the policy probability distribution $p_{RL}$ as presented in the paper, we sampled a batch of inputs (size of 64) and selected the best action of each sampled state based on their probability outcomes. These states (or inputs), actions, and weights become inputs to the pre-trained SEQ2SEQ and the pre-trained backward SEQ2SEQ model, and we obtain log probabilities from each model. Then, the mutual information score and the gradient in Eq.\ref{eq:gradient} are computed. According to the paper \cite{Li}, stochastic gradient descent (SGD) was applied to the objective function \ref{eq:obj_func}. However, since SGD is for minimization and the objective function has to be maximized, we used SGA instead to update variables of the encoder-decoder RNN. In this project, we implemented mutual information model without the curriculum and the baseline strategies.

\section{Results}
\subsection{DRL-SEQ2SEQ}
The training results of the backward SEQ2SEQ model after 109000 steps are presented in Table.\ref{table:backward_seq2seq}. Since it is a backward model, the bot generates the most likely utterance that will have the input utterance as its response. The examples in the Good section show good performance of the model. The generated responses in the Bad section do not make sense. The model tends to generate certain responses a lot such as "i am going to go to the bathroom", or to repeat all or some words in an input utterance. The training results of the SEQ2SEQ model with concatenated utterances after 107000 steps are presented in Table.\ref{table:seq2seq}. Each row of the table is one conversation, since the model consider two previous sentences. However, from the results, it was not hard to see that the model responses considering both two previous utterances. Also, it generates "i am sorry" or "no" many times. For example, for almost any human input utterance starting with "can you", it generates "no". \\
The results of the mutual information model were not good at all. First of all, since the mutual information score is averaged by the number of words in a sentence and each term is negative (log probability), we could see that the shorter a response is, the higher its score is. Also, it tends to generates lots of repeated words in a sentence. The example is shown in \ref{table:mi_01}. Therefore, we added an additional reward for the number of words, $N_a$ of an output sentence $a$ and penalties for the repeated words and for "i am sorry" or "no" response:
\begin{align}
    J(\theta) =&\; \mathbf{E} [-m(\hat{a},[u_{t-1},u_t])-c_l N_a \nonumber \\
     &\; + c_r \sum_{1\leq i < N_a} \mathbbm{1}(w_i^{(a)} == w_{i+1}^{(a)}) \nonumber \\ 
     &\; +c_d \mathbbm{1}(a == \text{"no" or "i am sorry"})]
\end{align}
where $c_l$, $c_r$, and $c_d$ are constant weights and $w_i^{(a)}$ is the $i$th word in a sentence $a$. However, those additional reward and penalties didn't improve the overall quality of responses. The Table.\ref{table:mi_02} shows an example of after 20 steps and the Table.\ref{table:mi_03} shows an example after training for 540 steps. Since we considered only the next word in the repeated-words penalty, the generated response repeat a set of words. Also, the outcomes tend to be unnecessarily long.

\begin{table}[bt!]
    \centering
    \small
    \caption{\small Backward Seq2Seq Model Examples}
    \begin{tabularx}{0.48\textwidth}{c|rl}
    	\hline
      \hline
        & \textbf{human:} & i do not know man\\
		& \textbf{bot:} & what do you think\\		
& \textbf{human:} & that s amazing \\
& \textbf{bot:} & wow \\
& \textbf{human:} & that works best for me \\
& \textbf{bot:} & i will be right back \\
Good & \textbf{human:} & I love you too \\
& \textbf{bot:} & i love you \\
& \textbf{human:} & i know i am hot \\
& \textbf{bot:} & i am a little nervous \\
& \textbf{human:} & i am done with you \\
& \textbf{bot:} & i am a fucking bitch \\
\hline
& \textbf{human:} & i am going to go to the bathroom\\
& \textbf{bot:} & i am going to go to the bathroom \\
& \textbf{human:} & this is weird \\
& \textbf{bot:} & i am going to be a great man \\
Bad & \textbf{human:} & okay\\
& \textbf{bot:} & i am going to go to the bathroom\\
& \textbf{human:} & you look gorgeous \\
& \textbf{bot:} & i am going to go see the kitchen \\
& \textbf{human:} & fine\\
& \textbf{bot:} & i am fine\\
      \hline
    \end{tabularx}
    \label{table:backward_seq2seq}
\end{table}

\begin{table}[bt!]
    \centering
    \small
    \caption{\small Seq2Seq Model Examples}
    \begin{tabularx}{0.48\textwidth}{rl}
    	\hline
      \hline
        \textbf{human:} & this is bad \\
	\textbf{bot:} & this is a joke \\
	\textbf{human:} & what joke \\
	\textbf{bot:} & i do not know \\
	\textbf{human:} & what are you doing \\
	\textbf{bot:} & nothing \\
	\hline
	\textbf{human:} & you are a loser\\
	\textbf{bot:} & you are a real joke you know \\
	\textbf{human:} & what is a real joke \\
	\textbf{bot:} & i do not know \\
	\hline
	\textbf{human:} & how old are you \\
	\textbf{bot:} & eleven \\
	\textbf{human:} & you are so young boy \\
	\textbf{bot:} & i am sorry \\
	\textbf{human:} & it's totally okay \\
	\textbf{bot:} & i am sorry \\
	\textbf{human:} & what are you up to this weekend \\
	\textbf{bot:} & i am sorry \\
	\hline
	\textbf{human:} & where are you from \\
	\textbf{bot:} & where are you going \\
	\textbf{human:} & i am going home \\ 
	\textbf{bot:} & i am going to get you out of here \\
	\textbf{human:} & thank you but where \\
	\textbf{bot:} &  what is this \\
	\textbf{human:} & nothing \\
	\textbf{bot:} & i am sorry \\
      \hline
    \end{tabularx}
    \label{table:seq2seq}
\end{table}

\begin{table}[bt!]
    \centering
    \small
    \caption{\small Mutual Information Model Results}
    \begin{tabularx}{0.48\textwidth}{X}
        \hline \hline
        The highest probability response among actions in one batch after 10 steps \\
        \hline    
        \textbf{bot:} i am flattered flattered truancy truancy truancy truancys truancy truancy truancy truancy truancy truancy truancy truancy truancy truancy truancy truancy truancy truancy truancy truancy truancy truancy  truancy soviets soviets soviets soviets soviets soviets soviets soviets soviets soviets \\
        \hline \hline
        The lowest probability response in one batch after 10 steps \\
        \hline
        \textbf{bot:} i am \\
        \hline \hline  
    \end{tabularx}
    \label{table:mi_01}
\end{table}
\begin{table}[bt!]
    \centering
    \small
    \caption{\small Mutual Information Model with additional reward/penalty}
    \begin{tabularx}{0.48\textwidth}{X}
        \hline \hline
        The highest probability response among actions in one batch after 20 steps \\
        \hline    
        \textbf{bot:} i have loveyou loveyou loveyou loveyou loveyou loveyou loveyou loveyou loveyou loveyou loveyou loveyou loveyou loveyou loveyou loveyou loveyou loveyou loveyou loveyou loveyou loveyou loveyou loveyou loveyou loveyou loveyou loveyou loveyou loveyou loveyou loveyou loveyou loveyou loveyou loveyou loveyou \\
        \hline \hline
        The lowest probability response in one batch after 10 steps \\
        \hline
        \textbf{bot:} yeah \\
        \hline \hline    
    \end{tabularx}
    \label{table:mi_02}
\end{table}

\begin{table}[bt!]
    \centering
    \small
    \caption{\small Mutual Information Model with additional reward/penalty after 540 steps}
    \begin{tabularx}{0.48\textwidth}{rX}
        \hline \hline
        \textbf{human:} & how old are you?    \\
        \textbf{bot:} & 000 toppy * didit pyramid * didit pyramid * didit pyramid * didit pyramid * didit pyramid * didit pyramid * didit pyramid * didit pyramid * didit pyramid * didit pyramid * didit pyramid * didit pyramid *\\
        \textbf{human:} & how are you doing? \\
        \textbf{bot:} & fiancee pyramid * didit * didit * didit * didit * didit * didit * didit * didit * didit * didit * didit * didit *\\
        \textbf{human:} & okay? \\
        \textbf{bot:} & fiancee pyramid * didit pyramid * didit pyramid * didit pyramid * didit pyramid * didit pyramid * didit pyramid * didit pyramid * didit pyramid * didit pyramid * didit pyramid * didit pyramid * didit pyramid * \\
        \hline \hline    
    \end{tabularx}
    \label{table:mi_03}
\end{table}


\subsection{HRED}

In this section we discuss the results for HRED. It was trained for 5 days on a GeForce GTX TITAN X GPU. The results after five hours are displayed in table 8. We keep the example brief and show an elucidating exmaple: after five hours the model appears to have acquired a basic grammar, but have a tendancy to output repeated words. Next we direct our attention to table 8, where we show the results after training for 72 hours. We observed a marked decrease in sentences with repeated words, and the table shows one example of a response that appears to have some ``semantic" relationship with the question asked. We ultimately trained the models for 5 days, and the results are displayed in table 10. In general, the results after 5 days do not appear qualitatively better than results after just 72 hours. In particular, we see no evidence that HRED is tracking ``some" conversation state. 

\begin{table}[bt!]
    \centering
    \small
    \caption{\small HRED after 5 hours}
    \begin{tabularx}{0.48\textwidth}{rX}
        \hline \hline
        \textbf{human:} & to be quite honest it is not very clear for me either?    \\
        \textbf{bot:} & what let s a  bitch and i m know you s gone black to you in you in you to you to you out of it to black to move to me *\\
        \hline \hline    
    \end{tabularx}
    \label{table:mi_03}
\end{table}

\begin{table}[bt!]
    \centering
    \small
    \caption{\small HRED after 5 hours}
    \begin{tabularx}{0.48\textwidth}{rX}
        \hline \hline
        \textbf{human:} & do you have any weapons explosives. any history of violent conduct \\
        \textbf{bot:} & a very interracial couples you re not gonna die you that s complicated \\
        \hline \hline    
    \end{tabularx}
    \label{table:mi_03}
\end{table}

\begin{table}[bt!]
    \centering
    \small
    \caption{\small HRED after 5 hours}
    \begin{tabularx}{0.48\textwidth}{rX}
        \hline \hline
        \textbf{human:} & i can t remember anything \\
        \textbf{bot:} & i got where the target and make preparation on froebe in jill \\
        \textbf{human:} & i m a guinea pig \\
        \textbf{bot: } & morning the problem your brutality \\
        \hline \hline    
    \end{tabularx}
    \label{table:mi_03}
\end{table}


\section{Discussion}
In this project, we implemented HRED and DRL-SEQ2SEQ model with the OpenSubtitle Dataset. We trained HRED for five days but saw no dramatic qualitative improvements after two day. Futhermore, there is no qualitative evidence that HRED is keeping track of conversational state. There are several possible reasons for this. First the data set may not have been appropriately preprocessed for our task, looking at the last three sentences of table 2, it is clear that the second to last two sentences are spoken by the same person. Future work would certainly need to be much more careful in their annotation of speakers if they choose to use this data set. Furthermore, since HRED is hypothesized to improve chatbot performance by keeping track of a session state, future data sets should ensure some session state exists by inspection. For example looking through the movie corpus, often times it is not clear there is such a state, especially if there are narrator interjections in the script, or scene cuts. Finally, the implementation of HRED may also limit the model's ability to learn full conversations. Since the current implementation of HRED assumes the entire session must be passed in as one vector, in practice we had to limit our sessions to four turns each, and each utterance could be at most 50 tokens long. This is a very stringent criteria in practice, since a conversation with explicit state, ie: greeting, middle and finally the good-bye is certainly longer than four turns. On a more general note, the notion that there could be enough data to place a distribution over the set of all conversations of arbitrary length may be unrealistic unless we are considering the most stringent definition of ``conversation" possible. Restricting HRED to a limited domain (ie by topic) may alleviate such issues. 

For the implementation of DRL-SEQ2SEQ, our mutual information model generated poor responses and we could see that the model quickly diverges to wrong outcomes during training. There are several possible reasons for this poor performance. First, we did not use the curriculum strategy, but the strategy may play a critical role in RL learning since the reward functions do not perfectly describe quality of responses. Second, we used 0.5 for the learning rate of SGD (the default value in the tensorflow seq2seq model), but it may have been too large because the performance changed very quickly. Lastly, not using the baseline strategy may also have affected the performance since it is known that the strategy reduces the variance of PGO. 
% include your own bib file like this:
%\bibliographystyle{acl}
%\bibliography{acl2017}
\bibliography{cis700}
\bibliographystyle{aaai}

\end{document}
